\documentclass{scrartcl}
\usepackage[de]{ukon-infie}

\Names{Name 1, Name 2}
\Lecture[Abk]{Fach - volle Länge}
\Term{WS 2011/2012}
\Fachbereich{Testfachbereich (optional)}

\begin{document}
    \begin{ukon-infie}[Abgabedatum]{42}

        \exercise{Venn Diagramme}{10}{
            \emph{Explain the differences, advantages and disadvantages between the following algorithms:}
            \begin{enumerate}
            %(a)
            \question{Draw some Venn diagramms with 2 sets.}{
            \Venn[blue,green]{1200}
            \Venn[blue][A,B,C]{1100}
            \Venn[blue,green]{1120}
            \Venn[blue,red,green][A,B,U]{1231}
            }
            %(b)
            \question{Draw some Venn diagramms with 3 sets.}{
            \bigVenn[blue]{10000000}
            \bigVenn[blue]{11000000}
            \bigVenn[blue]{11100000}
            \bigVenn[blue]{11110000}
            \bigVenn[blue][A,R,G,U]{11111000}
            \bigVenn[blue,green]{11211100}
            \bigVenn[blue,green,red,yellow][A,R,G,U]{11312410}
            \bigVenn[blue,green,red,yellow]{11213141}
            }
            \newpage %this newpage must be used inside exercises
            \question{Do some examples with some source code.}{
            \loadCpp{hello_world.cpp}
            \loadJava{hello_world}
            \loadSource{python}{hello_world.py}
            }
            \end{enumerate}
        }

        \newPage %this newPage must be used between exercises

	\additionalExercise{Do some colorfull text and symbol stuff.}{5.5}{
		\red{red text} \\
		\blue{blue text} \\
		\green{green Schrift} \\
		\yellow{yelllow text} \\
		\darkRed{dark red text} \\
		\darkBlue{dark blue text} \\
		\darkGreen{dark green text} \\
		\darkYellow{dark yellow text} \\
		\correct \\
		\wrong
	}

	\exerciseNumbering{Do something, where the questions are numbered}{2}{
		\begin{enumerate}
			\question{}{
				\red{do Something}
			}
			\question{}{
				\blue{do Something else}
			}
		\end{enumerate}
	}

    \end{ukon-infie}
\end{document}
